% start of file `template.tex'.
%% Copyright 2006-2013 Xavier Danaux (xdanaux@gmail.com).
%
% This work may be distributed and/or modified under the
% conditions of the LaTeX Project Public License version 1.3c,
% available at http://www.latex-project.org/lppl/.


\documentclass[11pt,a4paper,sans]{moderncv}        % possible options include font size ('10pt', '11pt' and '12pt'), paper size ('a4paper', 'letterpaper', 'a5paper', 'legalpaper', 'executivepaper' and 'landscape') and font family ('sans' and 'roman')

% moderncv themes
\moderncvstyle{classic}                             % style options are 'casual' (default), 'classic', 'oldstyle' and 'banking'
\moderncvcolor{blue}                               % color options 'blue' (default), 'orange', 'green', 'red', 'purple', 'grey' and 'black'
%\renewcommand{\familydefault}{\sfdefault}         % to set the default font; use '\sfdefault' for the default sans serif font, '\rmdefault' for the default roman one, or any tex font name
%\nopagenumbers{}                                  % uncomment to suppress automatic page numbering for CVs longer than one page

% character encoding
\usepackage[utf8]{inputenc}                       % if you are not using xelatex ou lualatex, replace by the encoding you are using
%\usepackage{CJKutf8}                              % if you need to use CJK to typeset your resume in Chinese, Japanese or Korean

% adjust the page margins
\usepackage[scale=0.75]{geometry}
%\setlength{\hintscolumnwidth}{3cm}                % if you want to change the width of the column with the dates
%\setlength{\makecvtitlenamewidth}{10cm}           % for the 'classic' style, if you want to force the width allocated to your name and avoid line breaks. be careful though, the length is normally calculated to avoid any overlap with your personal info; use this at your own typographical risks...

% personal data
\name{Luke}{Fraser}
\title{Computer Science Engineering}                               % optional, remove / comment the line if not wanted
\address{335 Vine St Vacaville, CA}{95688}{USA}% optional, remove / comment the line if not wanted; the "postcode city" and and "country" arguments can be omitted or provided empty
\phone[mobile]{(707)-592-1544}                   % optional, remove / comment the line if not wanted
%\phone[fixed]{}                    % optional, remove / comment the line if not wanted
%\phone[fax]{+3~(456)~789~012}                      % optional, remove / comment the line if not wanted
\email{Luke.Fraser.A@gmail.com}                               % optional, remove / comment the line if not wanted
%\homepage{www.johndoe.com}                         % optional, remove / comment the line if not wanted
%\extrainfo{additional information}                 % optional, remove / comment the line if not wanted
\photo[64pt][0.4pt]{picture}                       % optional, remove / comment the line if not wanted; '64pt' is the height the picture must be resized to, 0.4pt is the thickness of the frame around it (put it to 0pt for no frame) and 'picture' is the name of the picture file
%\quote{Some quote}                                 % optional, remove / comment the line if not wanted

% to show numerical labels in the bibliography (default is to show no labels); only useful if you make citations in your resume
%\makeatletter
%\renewcommand*{\bibliographyitemlabel}{\@biblabel{\arabic{enumiv}}}
%\makeatother
%\renewcommand*{\bibliographyitemlabel}{[\arabic{enumiv}]}% CONSIDER REPLACING THE ABOVE BY THIS

% bibliography with mutiple entries
%\usepackage{multibib}
%\newcites{book,misc}{{Books},{Others}}
%----------------------------------------------------------------------------------
%            content
%----------------------------------------------------------------------------------

% \setlength{\parindent}{0.5cm}
\begin{document}
%\begin{CJK*}{UTF8}{gbsn}                          % to typeset your resume in Chinese using CJK
%-----       resume       ---------------------------------------------------------
\makecvtitle



\section{Education}
\cventry{2009--2014}{BSE: Computer Science Engineering | Minor: Mathematics}{}{University Nevada, Reno}{}{Emphasis: Computer Vision}  % arguments 3 to 6 can be left empty
%\cventry{year--year}{Degree}{Institution}{City}{\textit{Grade}}{Description}

%\section{Master thesis}
%\cvitem{title}{\emph{Title}}
%\cvitem{supervisors}{Supervisors}
%\cvitem{description}{Short thesis abstract}

\section{Experience}
\subsection{Vocational}
\cventry{2012}{Engineering Presenter}{Department of Computer Science}{Reno}{}{Worked for the Mobile Engineering lab informing k-6 graders about engineering opportunities and teaching engineering skills.\newline{}}

\cventry{2013--Present}{Intern}{Bally Technologies Inc.}{Reno}{}{Working as a software engineering intern to help develop Class III Nevada style games.}

%\cventry{year--year}{Job title}{Employer}{City}{}{Description}

% \section{Languages}
% \cvitemwithcomment{Spanish}{Moderate}{More confident in written word than spoken}
%\cvitemwithcomment{Language 2}{Skill level}{Comment}

\section{About Me}
Luke Fraser attended Vacaville High School in Vacaville California, where he was given the great opportunity to participate in the FIRST Robotics competition. In the competition he was a lead designer and modeler underneath the president of the club, as well as one of the drivers for the actual competition. \vspace{.15cm}

His major is Computer Science Engineering, and he has taken courses in C++,  Java, Scheme, and Python. He is proficient in using many graphics design programs and 3D modeling programs as listed below. He is an intern at Bally Technologies where he works as a software engineer and develops Class III games. He also worked for the Engineering Outreach program, working in the community to present engineering to young children to raise awareness for the program and promote excitement about math and sciences among young children.\vspace{.15cm}

During his college career Luke has taken on many projects of his own. Since taking his first computer engineering  course  on  Micro-controllers (8051 and ARM 7), Luke has continued work with the assembly language and has worked on integration of computer parts with different micro-controllers(C8051F005). He has worked with OpenGL and Qt to develop a real-time game engine. This was inspired by a class offered at Brown University(cs195u). He is also working to create a 2D/3D fluid/gas dynamics computing system to simulate water and gases interacting with other objects. His most recent project is in computer vision. He is currently developing a system to monitor traffic intersection to determine when a vehicle has made a u-turn. All of his projects are influenced by his interest in 3D simulation, rendering, and Computer Vision.\\

\section{Technical skills}
\subsection{Languages}
\cvdoubleitem{-}{C/C++}{-}{Python}
\cvdoubleitem{-}{Assembly}{-}{Scheme}
\cvdoubleitem{-}{Java}{-}{C\#}
\cvdoubleitem{-}{Haskell}{-}{Linux Scripting}
\cvdoubleitem{-}{GNU Makefile}{}{}

\subsection{libraries/APIs}
\cvdoubleitem{-}{OpenGL}{-}{Qt}
\cvdoubleitem{-}{OpenCV}{-}{ROS-Hydro}
% \cvdoubleitem{Java}{}{C\#}{}

\subsection{Applications}
\cvdoubleitem{-}{SideFX Houdini}{-}{Photoshop}
\cvdoubleitem{-}{Solid Works}{-}{After Effects}
\cvdoubleitem{-}{Autodesk Maya}{-}{Autodesk 3DS Max}
\cvdoubleitem{-}{The Foundry Nukex}{-}{Mantra}
\cvdoubleitem{-}{Microsoft Office}{}{}

\section{Interests}
\cvitem{Computer Vision}{Computer Vision is a field in computer science that strives to understand real-world environments and events from images. There are many sub-fields of computer vision and I have covered a large gamut of computer vision, but my main focus has been with vehicle event recognition and tracking.}
\cvitem{Fluid Simulation}{Computation fluid dynamics is the implementation of the Navier Stokes equations to model incompressible/compressible fluids. These simulations are used to understand aerodynamics as well produce amazing visuals for cinematic movies. I have always been interested in dynamics simulation on computers and constantly look at new techniques for solving these simulations. I even work to develop my own simulators to test my understanding of new topics.}


%\cvlistitem{Item 1}
%\cvlistitem{Item 2}
%\cvlistitem{Item 3. This item is particularly long and therefore normally spans over several lines. Did you notice the indentation when the line wraps?}

% \section{Extra 2}
% \cvlistdoubleitem{Item 1}{Item 4}
% \cvlistdoubleitem{Item 2}{Item 5\cite{book1}}
% \cvlistdoubleitem{Item 3}{Item 6. Like item 3 in the single column list before, this item is particularly long to wrap over several lines.}

%\section{References}
%\begin{cvcolumns}
%  \cvcolumn{Category 1}{\begin{itemize}\item Person 1\item Person 2\item Person 3\end{itemize}}
%  \cvcolumn{Category 2}{Amongst others:\begin{itemize}\item Person 1, and\item Person 2\end{itemize}(more upon request)}
%  \cvcolumn[0.5]{All the rest \& some more}{\textit{That} person, and \textbf{those} also (all available upon request).}
%\end{cvcolumns}

% Publications from a BibTeX file without multibib
%  for numerical labels: \renewcommand{\bibliographyitemlabel}{\@biblabel{\arabic{enumiv}}}% CONSIDER MERGING WITH PREAMBLE PART
%  to redefine the heading string ("Publications"): \renewcommand{\refname}{Articles}
\nocite{*}
\bibliographystyle{plain}
\bibliography{publications}                        % 'publications' is the name of a BibTeX file

% Publications from a BibTeX file using the multibib package
%\section{Publications}
%\nocitebook{book1,book2}
%\bibliographystylebook{plain}
%\bibliographybook{publications}                   % 'publications' is the name of a BibTeX file
%\nocitemisc{misc1,misc2,misc3}
%\bibliographystylemisc{plain}
%\bibliographymisc{publications}                   % 'publications' is the name of a BibTeX file

\clearpage
%-----       letter       ---------------------------------------------------------
% recipient data
\iffalse

\recipient{Company Recruitment team}{Company, Inc.\\123 somestreet\\some city}
\date{January 01, 1984}
\opening{Dear Sir or Madam,}
\closing{Yours faithfully,}
\enclosure[Attached]{curriculum vit\ae{}}          % use an optional argument to use a string other than "Enclosure", or redefine \enclname
\makelettertitle

Lorem ipsum dolor sit amet, consectetur adipiscing elit. Duis ullamcorper neque sit amet lectus facilisis sed luctus nisl iaculis. Vivamus at neque arcu, sed tempor quam. Curabitur pharetra tincidunt tincidunt. Morbi volutpat feugiat mauris, quis tempor neque vehicula volutpat. Duis tristique justo vel massa fermentum accumsan. Mauris ante elit, feugiat vestibulum tempor eget, eleifend ac ipsum. Donec scelerisque lobortis ipsum eu vestibulum. Pellentesque vel massa at felis accumsan rhoncus.

Suspendisse commodo, massa eu congue tincidunt, elit mauris pellentesque orci, cursus tempor odio nisl euismod augue. Aliquam adipiscing nibh ut odio sodales et pulvinar tortor laoreet. Mauris a accumsan ligula. Class aptent taciti sociosqu ad litora torquent per conubia nostra, per inceptos himenaeos. Suspendisse vulputate sem vehicula ipsum varius nec tempus dui dapibus. Phasellus et est urna, ut auctor erat. Sed tincidunt odio id odio aliquam mattis. Donec sapien nulla, feugiat eget adipiscing sit amet, lacinia ut dolor. Phasellus tincidunt, leo a fringilla consectetur, felis diam aliquam urna, vitae aliquet lectus orci nec velit. Vivamus dapibus varius blandit.

Duis sit amet magna ante, at sodales diam. Aenean consectetur porta risus et sagittis. Ut interdum, enim varius pellentesque tincidunt, magna libero sodales tortor, ut fermentum nunc metus a ante. Vivamus odio leo, tincidunt eu luctus ut, sollicitudin sit amet metus. Nunc sed orci lectus. Ut sodales magna sed velit volutpat sit amet pulvinar diam venenatis.

Albert Einstein discovered that $e=mc^2$ in 1905.

\[ e=\lim_{n \to \infty} \left(1+\frac{1}{n}\right)^n \]

\makeletterclosing
\fi

%\clearpage\end{CJK*}                              % if you are typesetting your resume in Chinese using CJK; the \clearpage is required for fancyhdr to work correctly with CJK, though it kills the page numbering by making \lastpage undefined
\end{document}


%% end of file `template.tex'.
